\documentclass{article}
\usepackage[utf8]{inputenc}
\usepackage[T1]{fontenc}
\usepackage[a4paper,margin=1cm]{geometry}

\usepackage[ngerman]{babel}

\title{Rollenspiel-Selbsteinschätzung}
\author{Rollenspiel: Geschichte, Theorie und Praxis}
\date{WinterAkademie 2015/16}

\newcommand\frage[1]{\section{Frage}\label{#1}}

\begin{document}

\maketitle

\textit{Kursivtext sind Ereignisse, die sich in der Fiktion abspielen}. Normaltext sind Ereignisse, die sich in der Realität Ereignissen.

\frage{hosendiebstahl}
Die Abenteurergruppe sitzt in einer Taverne und befragt Anwesende
bezüglich eines Mordfalls. Einer der Abenteurer, ein Dieb, stiehlt
zunächst den Geldbeutel, dann  Dolch und letztlich sogar die Hose
eines Adligen, der an der Bar steht.

\begin{itemize}
\item  Wärest du gerne dabei gewesen?
\item  Wärest du gerne der Protagonist gewesen?
\item  Gibt es etwas, was dich an dieser Szene stört?
  \begin{itemize}
  \item Nein, das klingt toll.
  \item Ja, denn kein Dieb würde soetwas tun.
  \item Ja, denn Abenteurer sind per se unplausibel.
  \item Ja, denn ein mittelalterliches Fantsasy-Setting gefällt mir nicht.
  \item Ja, denn das ist unmöglich.
  \item Ja, denn ich will keine komplizierten Ermittlungen führen.
  \item Ja, denn der Diebstahl behindert die Ermittlungen, anstatt zu helfen.
  \item Ja, denn ein Dieb würde sich niemals trauen, einen Adligen so zu bestehlen.
  \end{itemize}
\end{itemize}

\frage{schlachtengetuemmel}
\textit{Die Helden sind an einem Kampf gegen die Orkhorden des
  finsteren Zauberers vom schwarzen Turm.} Die Spieler klären zunächst
die Lage des Schlachtfeldes und mögliche strategische
Operationen. \textit{Dann beginnt der Kampf. Der mutige Krieger führt
  die Offensive in die Flanke des feindlchen Heeres und besiegt viele
  Gegner im heldenhaften Zweikampf}. Der Spieler würfelt viele
Angriffswürfe und hat großes Glück. Nach etwa einer Stunde Spielzeit
endet die Szene mit dem Sieg der Helden. Am Ende feiern die Spieler
ausgelassen den Sieg und beglückwünschen den Spieler des Kriegers.

\begin{itemize}
\item  Wärest du gerne dabei gewesen?
\item  Wärest du gerne der Protagonist gewesen?
\item  Gibt es etwas, was dich an dieser Szene stört?
  \begin{itemize}
  \item Nein, das klingt toll.
  \item Ja, denn das ist schreckliche klischeehaft.
  \item Ja, denn wo sollen diese Orkhorden überhaupt herkommen?
  \item Ja, denn spätestens der zehnte Ork würde flüchten, statt sich im Zweikampf zu stellen.
  \item Ja, denn die Szene dauert zu lange.
  \item Ja, denn ein einzelner Spieler bekommt zu viel Rampenlicht.
  \item Ja, denn der Glücksfaktor ist zu groß.
  \item Ja, denn den Kampf auszuspielen bringt die Geschichte nicht weiter.
  \item Ja, denn strategische Operationen passen nicht zu Heldengeschichten.
  \item Ja, denn in einer Schlacht gibt es keine Zweikämpfe.
  \item Ja, denn Orks und Zauberer gibt es nicht.
  \item Ja, denn der Dieb von Frage \ref{hosendiebstahl} hat nichts zu tun.
  \item Ja, denn ein einzelner Krieger könnte niemals dutzende Gegner besiegen.
  \item Ja, denn den Gegnern wird keinerlei Individualität zugestanden. Sie sind nur Kanonenfutter.
  \end{itemize}
\end{itemize}

\section{Raumschlacht}
\label{raumschlacht}

Der Pilot (eine Spielerfigur) befiehlt, dem Waffenoffizier des
Schiffes (eine weitere Spielerfigur), die Laserkanonen ein letztes Mal
abzufeuern, bevor das verfolgte Raumschiff den Laserschüssen der
Verfolger entkommt, indem es einen Sprung durch ein Wurmloch
ausführt. Der Austrittspunkt ist 27.3 Parsec entfernt. Die Laser
entschwinden zischend in die unendlichen Weiten des Raumes.

\begin{itemize}
\item  Wärest du gerne dabei gewesen?
\item  Wärest du gerne an Bord gewesen?
\item  Gibt es etwas, was dich an dieser Szene stört?
  \begin{itemize}
  \item Nein, das klingt toll.
  \item Ja, denn Parsec ist keine SI-Einheit.
  \item Ja, denn dieser Satz ist zu lang.
  \item Ja, denn woher sollte der Pilot wissen, wie weit der Austrittspunkt entfernt liegt.
  \item Ja, denn eine Spielerfigur hat hier Befehlsgewalt über eine andere Spielerfigur.
  \item Ja, denn ein Pilot gibt keine Befehle, sondern der Kommandant.
  \item Ja, denn das ist mir viel zu wissenschaftlich.
  \item Ja, denn die Geschichte widerspricht unserem aktuellen Verständnis der Physik.
  \item Ja, denn Raumschiffkämpfe machen an sich keinen Sinn.
  \item Ja, denn im Weltraum gibt es keinen Ton.
  \item Ja, denn ich habe keine Lust, mit Leuten über Fragen wie die obigen zu diskutieren.
  \end{itemize}
\end{itemize}


\frage{gutundboese} Berlin, September 1945: Die Agenten, Spione im
Dienst ihrer Majestät, nähern sich dem Führerbunker in zivil und
werden dabei von einer Gruppe Kindersoldaten der Wehrmacht
überrascht. Eines der Kinder richtet seine HK G3 auf die Agenten,
zögert aber. Einer der Agenten eröffnet das Feuer.

\begin{itemize}
\item  Wärest du gerne auf der Seite der Agenten mitgespielt?
\item  Wärest du gerne auf der Seite der Wehrmacht mitgespielt?
\item  Wärest du gerne der Schütze gewesen?
\item  Gibt es etwas, was dich an dieser Szene stört?
  \begin{itemize}
  \item Nein, das klingt toll.
  \item Ja, ich möchte nicht schon wieder über den zweiten Weltkrieg hören.
  \item Ja, denn ich möchte micht nicht mit so düsteren Themen beschäftigen.
  \item Ja, denn ich möchte keine Spiele in der realen Welt spielen?
  \item Ja, denn die Spieler sollten immer Helden spielen.
  \item Ja, denn ich möchte keine Szenarien spielen, in denen Menschen sterben.
  \item Ja, denn ich möchte keine Szenarien spielen, in denen Kinder sterben.
  \item Ja, denn die Szene ist unplausibel. Der Krieg in Europa war bereits im Mai 1945 vorbei.
  \item Ja, denn mich interessiert der Typ des Gewehrs nicht.
  \item Ja, denn der Hckler und Koch G3 wurde zu diesen Zeitpunkt noch nicht hergestellt.
  \end{itemize}
\end{itemize}

\frage{steampunk}

Die Crew des Luftschiffs HMAS Elisabeth fliegt durch umkämpften
Luftraum auf dem Weg in die Kolonien mit einer kritischen Ladung
Gelbfieber-Impfstoff. Agenten des machthungrigen Zaren Grygory
Petrakov VII. haben sich unter die Crew gemischt und eine Meuterei
angezettelt. Der Kapitän beauftragt seine letzten loyalen Matrosen
(die Spielergruppe) damit, die Meuterei aufzulösen. Im Gespräch mit
den feindlichen Agenten lässt sich eine der Spielerfiguren mit der
Aussicht auf eine hochrangige Position im Zarenreich bestechen und
ermordet den ahnungslosen Kapitän.
\begin{itemize}
\item  Wärest du gerne dabei gewesen?
\item  Wärest du gerne der Überläufer gewesen?
\item  Was denkst du über den Spieler des Überläufers?
  \begin{itemize}
  \item Das ist ein Beispiel für exzellentes Rollenspiel!
  \item Das ist ein Beispiel für einen Spieler, der zu viel Aufmerksamkeit braucht.
  \item Das ist ein Beispiel für einen Spieler, der seine eigene Figur über das Interesse der Gruppe stellt.
  \end{itemize}
\item  Gibt es etwas, was dich an dieser Szene stört?
  \begin{itemize}
  \item Nein, das klingt toll.
  \item Ja, denn das behindert die Agenda der anderen Spielerfiguren.
  \item Ja, denn es stört den Verlauf einer Heldengeschichte.
  \item Ja, denn alle Spieler sollten Helden spielen.
  \item Ja, denn die Spieler sollten ihre Figuren nicht gegeneinander agieren lassen.
  \item Ja. Warum ein Zeppelin?
  \item Ja, denn ``Her Majestys Airship'' und ``Luftschiff'' sind redundant.
  \item Ja, denn der Spielleiter hätte dieses Verhalten nicht erlauben sollen.
  \end{itemize}
\end{itemize}

\frage{gefaengnis}

\textit{Nach langer Vorgeschichte waren die Investigatoren zur
  falschen Zeit am falschen Ort und sind aufgrund ungeschickten
  Verhaltens einer Spielerfigur jetzt im Gefängnis. Sie wurden wegen
  des Mordes verurteilt, den sie eigentlich untersuchen wollten.} Die
Spieler und der Spielleiter sprechen über die Situation. Die Spieler
sind überzeugt, dass es keinen Ausweg mehr für sie gibt und fragen, ob
die Gruppe das nächste mal eine neue Kampagne beginnen. Der
Spielleiter wiegelt ab und teilt mit, dass er ein paar Ideen hat, die
Situation zu lösen. \textit{Obwohl sie sich in der Stadt bereits
  genügend Feinde und keine Freunde gemacht haben, erscheint in
  letzter Sekunde vor der Exekution eine zwielichtige Gestalt, die
  ihnen einen Ausweg aus der Misere anbietet. Es stellt sich heraus
  dass sie zu einem Geheimdienst gehört, von dessen Existenz die
  Spieler bisher noch nichts wussten.}
\begin{itemize}
\item  Wärest du gerne dabei gewesen?
\item  Wärest du gerne der Schuldige für das Versagen gewesen?
\item  Gibt es etwas, was dich an dieser Szene stört?
  \begin{itemize}
  \item Nein, das klingt toll.
  \item Ja, denn nur die Spielerfigur des schuldigen Spielers sollte bestraft werden.
  \item Ja, denn die Spielerfiguren sollten mit den Bösewichten kämpfen, nicht mit den Autoritäten.
  \item Ja, denn nach langer Vorgeschichte sollten einzelne Dummheiten keine so drastischen Konsequenzen haben können.
  \item Ja, denn der Spielleiter sollte nicht vom vorgegebenen Skript abweichen, um die Spielerfiguren zu retten.
  \item Ja, denn der Spielleiter sollte sich mit den Spielern nicht über den weiteren Verlauf der Geschichte unterhalten.
  \item Ja, denn der Spielleiter hätte den erfahrenen Investigator von seinem Handeln abbringen müssen.
  \item Ja, denn ich möchte viel lieber Abenteuer erleben als im Justizsystem zu versauern.
  \item Ja, denn der Ausweg, den der Spielleiter erschafft, erscheint konstruiert.
  \end{itemize}
\end{itemize}

\frage{wettstreit}

Zwei Männer vom Feenvolk wetteifern um die Liebe einer
Menschenfrau. Die Frau fordert beide auf, ein Gedicht
vorzutragen. Beide Gedichte werden privat vorgetragen. Einige
Spielercharaktere befinden sich auf dem Fest und nehmen nicht an der
Szene teil. Dann entscheidet die Frau, welches Gedicht ihr besser
gefällt.

\begin{itemize}
\item  Wärest du gerne dabei gewesen?
\item  Wärest du gerne der Spieler des Gewinners gewesen?
\item  Wärest du gerne der Spieler des Verlierers gewesen?
\item  Wärest du gerne der Spieler der Frau gewesen?
\item  Gibt es etwas, was dich an dieser Szene stört?
  \begin{itemize}
  \item Nein, das klingt toll.
  \item Nein, nicht solange nur eine der beteiligten Personen eine Spielerfigur ist. 
  \item Nein, solange nicht beide Feenmänner Spielerfiguren sind,
    klingt das gut. Ich würde nicht wollen, dass Spieler sich miteinander messen müssen.
  \item Nein, solange die Menschenfrau keine Spielerfigur
    ist. Liebesbeziehungen zwischen Spielerfiguren sind für mich kein
    Bestandteil eines guten Spiels.
  \item Nein, solange nicht alle drei Spielerfiguren
    sind. Spielerfiguren sollten nicht andere Spielerfiguren werten.
  \item Nein, solange die Qualität des Gedichts nur durch Spielmechanik bestimmt wird.
  \item Nein, solange die Gedichte wirklich in der Runde vorgetragen werden.
  \item Ja, Gedichte interessieren mich nicht.
  \item Ja, wo ist die Heldengeschichte? Das sind doch nur unwichtige Techtelmechtel.
  \item Ja, es wird nicht gekämpft.
  \item Ja, was hat das noch mit Rollenspiel zu tun?
  \item Ja, Feen sollen sich nicht mit Menschen mischen.
  \item Ja, das ist Sodomie.
  \end{itemize}
\end{itemize}

exploration: karte

closure


immersionsfragen

großes risiko

teamwork

catharsis




handouts/kinesis

cairosis

ludus: charaktererschaffung

naches (?):

paida

schadenfreude

sociability (?)

venting

horror

pseudohistorisch 

\end{document}


